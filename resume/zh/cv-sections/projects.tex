\section{參與專案}

%%% ----- Best way to write items (Credit - FAANGPath)
        % \item Achieved X\% growth for XYZ using A, B, and C skills.
        % \item Led XYZ which led to X\% of improvement in ABC
        % \item Developed XYZ that did A, B, and C using X, Y, and Z.
        

%%%%%%% ----------------------------------- Sentiment Analysis ----------------------------------- %%%%%%%
\subsection*{AIdea Movie Review Sentiment Analysis \hfill June 2021 --- Present} 
\begin{zitemize}
    \item 在public leaderboard取得\textbf{Top-1 F-1 score (99.286\%)}成績。
    \item 應用MLOps框架DVC到專案上,管理資料預處理流程、模型版本迭代。
    \item 實驗不同特徵、模型對於文本情感預測的影響,實作詞袋模型、深度學習模型。並利用\textbf{BERT}集成模型取得最高成績。
\end{zitemize}


%%%%%%% ----------------------------------- Impactio ----------------------------------- %%%%%%%
\subsection*{Impactio \hfill 八月 2020 --- 三月 2021} 
    \begin{zitemize}
        \item 利用\textbf{GraphQL}開發針對學術人才的社交網絡應用。
        \item 優化GraphQL built-in分頁方法,提升約\textbf{10\%效能},並解決深度分頁問題。
        \item 採用非同步方法優化資料處理流程,提升約\textbf{30\%效率}。
        \item 利用\textbf{Dask}開發Google學術分散式爬蟲系統。
    \end{zitemize}


%%%%%%% ----------------------------------- POS Tagging and Dependency Parsing ----------------------------------- %%%%%%%
\subsection*{Part-of-Speech Tagging and Dependency Parsing \hfill 十月 2018 --- 五月 2019} 
    \begin{zitemize}
        \item 聯合詞性標註、依存句法分析模型在Universal Dependency 2.0數據集上得到\textbf{11\%LAS準確度提升}。
        \item 實驗結果證明詞性對於依存句法分析的重要性,再去掉詞性特徵後會導致依存句法分析模型\textbf{降低近10\%的準確度}。
    \end{zitemize}


%%%%%%% ----------------------------------- Joint Representations of Knowledge Entities and Texts ----------------------------------- %%%%%%%
\subsection*{Joint Representations of Knowledge Entities and Texts \hfill 一月 2018 --- 十二月 2018} 
    \begin{zitemize}
        \item 聯合表示學習模型在SemEval-2017跨語言文本相似度計算任務上取得\textbf{10\%的皮爾森相關性提升}。
        \item 利用\textbf{遠程監督方法}在維基百科上自動建立雙語對照語料,並貢獻數據集。
        \item 提出跨語言信息檢索任務用以評測跨語言文本相似度,相比於其他跨語言方法取得近\textbf{30\%Top-10準確度的提升}。
    \end{zitemize}


%%%%%%% ----------------------------------- News Miner ----------------------------------- %%%%%%%
\subsection*{News Miner \hfill 九月 2016 --- 九月 2017} 
    \begin{zitemize}
        \item 開發新聞熱點分析應用,設計增量式聚類方法。將時間複雜度從\textbf{平方級降低到線性複雜度}。
        \item 導入基於向量的文本相似度計算方法,優化新聞文本搜尋結果。
        \item 利用循環神經網絡設計依存句法分析系統,平均每分鐘可以處理約1K+的句子。
    \end{zitemize}
